\documentclass[12pt,a4paper,headings=optiontohead]{scrbook}
\usepackage[utf8]{inputenc}
\usepackage[italian]{babel}
\usepackage[margin=1.8cm,bottom=7em]{geometry}
\usepackage[subpreambles=false]{standalone}
\usepackage{amsmath}
\usepackage{amssymb}
\usepackage{amsthm} 
\usepackage{cancel}
\usepackage{graphicx}
\usepackage{mathtools}
\usepackage{float}
\usepackage{enumitem}
\usepackage{algorithm}
\usepackage{algorithmic}


\usepackage[normalem]{ulem}

\usepackage{blkarray}% http://ctan.org/pkg/blkarray

\newcommand{\matindex}[1]{\mbox{\scriptsize#1}}% Matrix index

\renewcommand{\qedsymbol}{\rule{0.7em}{0.7em}}
\DeclarePairedDelimiter{\abs}{\lvert}{\rvert}
\newcommand{\inter}{\begin{matrix}\prod\end{matrix}}
\newcommand{\verteq}{\rotatebox{90}{$\,=$}}
\newcommand{\equalto}[2]{\underset{\scriptstyle\overset{\mkern4mu\verteq}{#2}}{#1}}
\DeclarePairedDelimiter{\norma}{\lVert}{\rVert}
\newtheorem*{esempio}{Esempio}

\usepackage{import}
\usepackage{hyperref}
\begin{document}

%------------------------------------------------------------------------------------------------------
%------------------------------------------------------------------------------------------------------
%-----------------------------------------------INTESTAZIONE-------------------------------------------
%------------------------------------------------------------------------------------------------------
%------------------------------------------------------------------------------------------------------

\begin{titlepage}

\newcommand{\HRule}{\rule{\linewidth}{0.5mm}} % Defines a new command for the horizontal lines, change thickness here

\center % Center everything on the page
 
%----------------------------------------------------------------------------------------
%	HEADING SECTIONS
%----------------------------------------------------------------------------------------




\includegraphics[height=4cm]{../img/btllogo.jpg}\\[0.3cm]
%----------------------------------------------------------------------------------------
%	TITLE SECTION
%----------------------------------------------------------------------------------------

\HRule \\[0.4cm]
{ \huge \bfseries Relazione del progetto di}\\
{ \huge \bfseries Tecnologie Web\\[0.15 cm]} % Title of your document
\HRule \\[1.5cm]
 
%----------------------------------------------------------------------------------------
%	AUTHOR SECTION
%----------------------------------------------------------------------------------------
\emph{\Large{Autori:}}\\
\renewcommand{\arraystretch}{1.4}
 \begin{center}
 \begin{tabular}{r|l}	
	\textbf{Nominativo} & \textbf{Matricola}\\ \hline
Antonio \textsc{Badan} & 1201209\\
Simone \textsc{De Renzis} & 1187510\\
Francesco \textsc{Trolese} & 1187224\\
Luca \textsc{Veronese} & 1187571\\
 \end{tabular}
 \end{center}

%----------------------------------------------------------------------------------------
%	DATE SECTION
% %----------------------------------------------------------------------------------------
\vspace{1.5cm}
{\LARGE Università degli Studi di Padova}\\[0.4cm] % Name of your university/college
\textsc{\large{Dipartimento di Matematica}}\\[0.05cm]
\textsc{\large{Corso di Laurea in Informatica}}\\[0.5cm]% Include a department/university logo - this will require the graphicx package
{\Large Anno accademico 2020 - 2021}\\ % Date, change the \today to a set date if you want to be precise

\vfill % Fill the rest of the page with whitespace



\emph{\Large{Utenti preregistrati:}}\\

\renewcommand{\arraystretch}{1.4}
 \begin{center}
 \begin{tabular}{|r|l|l|l|}
 \hline
\textbf{Utente} & \textbf{mail} & \textbf{Password} & \textbf{Categoria}  \\ \hline \hline
 &  &  & \\ \hline

 \end{tabular}
 \end{center}

\end{titlepage}


\begin{center}
\pagebreak

\section*{Premessa}
\begin{minipage}{0.9\textwidth} 
\large{Relazione del progetto di Tecnologie Web, riguardante un sito di recensioni di libri, denominato \textit{"Between the Lines"}}

\end{minipage}

\end{center}
\pagebreak

\tableofcontents

% ############ CAPITOLO 1 ##################
\chapter{Sistema floating-point e propagazione degli errori; costo computazionale}
\chaptermark{Floating-point, errori, costo computazionale}

\end{document}
